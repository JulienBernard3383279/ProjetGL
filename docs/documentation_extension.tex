\documentclass[a4paper]{article}
\usepackage[utf8x]{inputenc}
\RequirePackage[utf8]{inputenc}

\begin{document}
\title{Compilateur Deca : Documentation de l'extension : Optimisation}
\author{\'Equipe 58}
\maketitle
\section{Introduction}
Pour notre compilateur Decac nous avons dû choisir une extension parmis plusieurs extension, et notre choix s'est porté sur l'optimisation. L'optimisation consiste à rendre un code plus performant, c'est à dire à l'exécuter plus rapidement en faisant moins d'opérations au niveau du processeur, ou des opérations qui sont plus rapides.
\section{Spécification de l'extension Optimisation}
\subsection{Optimisations implémentées}
Voici la liste des différentes spécifications de l'optimisation:
\begin{itemize}
\item \texttt{Dead Store} \\
      \texttt{decac -o1} \\
Le dead store consiste à éliminer les variables qui ne sont pas utiles dans le code.
\item \texttt{Constant Folding}\\
\texttt{decac -o2}\\
Le constant folding est une optimisation qui calcul tout les calculs constants et stocke les résultats au lieu de stocker le calcul. A l'exécution le processeur n'a plus besoin de faire les opérations, il à déjà le résultat en mémoire.
     \end{itemize}
\subsection{Optimisations envisagées}
\begin{itemize}
\item \texttt{Inlining} \\
Le inlining est une optimisation qui consiste à remplacer les appels des méthodes par le corps des méthodes, dans le cas ou la méthode ne prend pas trop de place en mémoire.
     \end{itemize}
\section{Analyse bibliographique}
\section{Choix de conception}
\section{Méthode de validation}
\section{Résultat}
\section{Améliorations possibles de l'extension}

\end{document}
