<<<<<<< HEAD
\documentclass[a4paper]{article}
\usepackage[utf8x]{inputenc}

\begin{document}
\title{Compilateur Deca : Manuel Utilisateur}
\author{\'Equipe 58}
=======
\documentclass[a4paper, 10pt, french]{article}
\usepackage[utf8]{inputenc}
\usepackage[margin=1.2in]{geometry}
\RequirePackage[utf8]{inputenc}


\title{Compilateur Decac : Manuel Utilisateur}
\author{\'Equipe gl58}

\begin{document}

>>>>>>> 52a438b7e889549a76e1bf24a4b6661733ba1f5f
\maketitle
\section{Introduction au compilateur}
Le compilateur Decac prend en argument un fichier source en Deca et le compile pour donner un fichier en code assembleur. \newline
Ce compilateur gère toutes les étapes de la compilation, c'est-à-dire le parsing (analyse lexical et syntaxique) du fichier source, l'analyse contextuelle du code et la génération du code assembleur.Il prend aussi en charge un ensemble d'options spécifiées dans ce manuel. 
\section{Ex\'ecution de decac}
Pour lancer le compilateur, il faut utiliser la commande \texttt{decac}. Cette commande possède différentes options d'exécution:
\begin{itemize}
    \item \texttt{decac}
    \newline
    Affiche les options possibles.
    \item \texttt{decac -b}
    \newline
    Affiche une banni\`ere de l'\'equipe qui a programm\'e ce compilateur.
    \item \texttt{decac -p <ficher deca>} \newline
    Parse le fichier pass\'e en argument et affiche sa decompilation.
    \item \texttt{decac -v <fichier deca>}
    \newline
    V\'erifie contextuellement le fichier deca. Affiche une erreur si il y en a.
    \item \texttt{decac -n <fichier deca>}
    \newline
    Exécute le fichier deca en ignorant les erreurs de débordement \`a l'exécution.
    \item \texttt{decac -r X <fichier deca>}
    \newline
    Execute le fichier deca en limitant les registres disponible \`a \texttt{R\{0\}...R\{X\}}.
    \item \texttt{decac -d <fichier deca>}
    \newline
    Active les traces de debug lors de l'exécution.
    \item{\texttt{decac -P <fichier(s) deca>}}
    \newline
    Si il y a plusieurs fichiers sources, les exécutent en parallèle.
    \item{\texttt{decac -o1 <fichier(s) deca>}}
    \newline
    Effectue l'extension deadstore sur le programme deca.
    \item{\texttt{decac -o2 <fichier(s) deca>}}
    \newline
    Effectue l'extension ConstantFolding sur le programme deca.

\end{itemize}
Plusieurs options peuvent être appelées simultanément, \texttt{-p} et \texttt{-v} exclus.
Le fichier \texttt{.ass} obtenu en sorti du compilateur peut \^etre pass\'e en exécutable avec la commande \texttt{ima <fichier assembleur>}.

\section{Erreurs levées par le compilateur}
\subsection{Erreurs lexicales}
\subsection{Erreurs syntaxiques}
\subsection{Erreurs contextuelles}
Voici les erreurs qui peuvent être levées par l'analyse contextuelle :
\begin{itemize}
\item \texttt{type or class undefined} \newline
Est levée si un identifier de type est attendu, mais l'identifier trouvé n'est pas un type prédéfini ou une classe qui a été définie.
\item \texttt{class extends type}
\newline
Est levée si, lors de la déclaration d'une classe, l'identifier suivant \texttt{extends} est un type (\texttt{int,float,boolean,void});
\item \texttt{class already defined}
\newline
Est levée si une même classe est déclarée plusieurs fois.
\item \texttt{field already defined}
\newline
Est levée si un champ est déclaré plusieurs fois ou déclaré avec le même nom qu'une méthode existante.
\item \texttt{field cannot be void}
\newline
Est levée si un champ a pour type \texttt{void}.
\item \texttt{method already defined}
\newline
Est levée si une méthode est déclarée plusieurs fois dans la même classe ou déclarée avec le même nom qu'un champ existant.
\item \texttt{method overrides method with different signature}
\newline
Est levée si une méthode essaie d'écraser une méthode avec une différente signature (différent type, différent nombre de paramètres ou diff\'erent types de paramètres).
\item \texttt{parameter cannot be void}
\newline
Est levée si un paramètre d'une méthode a pour type \texttt{void}.
\item \texttt{parameter already defined}
\newline
Est levée si plusieurs paramètres d'une même méthode ont le même nom.
\item \texttt{variable already defined}
\newline
Est levée si une variable est déclarée plusieurs fois.
\item \texttt{variable cannot be void}
\newline
Est levée si une variable a pour type \texttt{void}.
\item \texttt{return called in void method}
\newline 
Est levée si \texttt{return} est appelée dans une méthode qui a pour type \texttt{void}.
\item \texttt{type of expression must match method type}
\newline
Est levée si l'expression renvoyée n'est pas du même type que la méthode.
\item \texttt{expected type found class}
\newline
Est levée si l'expression est une instance de classe alors qu'une expression de type \texttt{int,float} ou \texttt{boolean} est attendue.
\item \texttt{incompatible class type}
\newline
Est levée si la classe attendue n'est pas une super-classe de la classe renvoyé. Ce cas est le seul qui permet d'initier l'instance d'une classe avec une différente classe.
\item \texttt{expected different type}
\newline
Est levée si l'expression trouvée n'est pas du type attendu, sauf si l'expression est de type \texttt{int} et le type attendu est \texttt{float}. Dans ce dernier cas, on converti l'\texttt{int} en \texttt{float}.
\item \texttt{only string, int and float expressions can be printed}
\newline
Est levée si un \texttt{void},\texttt{null} ou une instance de classe est mis en paramètre de \texttt{print} ou \texttt{println}.
\item \texttt{condition must be boolean}
\newline
Est levée si la condition dans un \texttt{if} ou \texttt{while} n'est pas booléen.
\item \texttt{operands must be int or float}
\newline
Est levée si les opérandes d'un opération arithmétique (ou d'une comparaison) ne sont pas numériques.
\item \texttt{operand must be int or float}
\newline
Est levée si l'opérande du moins unaire n'est pas numérique.
\item \texttt{operands must be boolean}
\newline
Est levée si les opérandes d'une opération booléenne n'est pas \texttt{boolean}.
\item \texttt{operand must be boolean}
\newline
Est levée si l'opérande de \texttt{Not(!)} n'est pas booléen.
\item \texttt{operands must have same type}
\newline
Est levée si les operandes d'une comparaison n'ont pas le meme type. Notamment, on ne peut pas comparer un \texttt{int} et un \texttt{float}.
\item \texttt{incompatible types for cast}
\newline
Est levée si expression est \texttt{Cast} dans un type incompatible a son propre type. Le seul \texttt{Cast} de type accepté est le passage de \texttt{int} \`a \texttt{float}.
\item \texttt{type cast to class}
\newline
Est levée si une expression de type \texttt{int,float} ou \texttt{boolean} est \texttt{Cast} dans une \texttt{class}.
\item \texttt{class cast to type}
\newline
Est levée si l'instance d'une classe est \texttt{Cast} en \texttt{int,float} ou \texttt{boolean}.
\item \texttt{classes incompatible for cast}
\newline
Est levée si l'instance d'une classe est \texttt{Cast} dans une classe incompatible. Cela se produit quand les deux classes ne sont pas dans la même hiérarchie de classes.
\item \texttt{identifier is not a class}
\newline
Est levée si l'identifier suivant \texttt{new} n'est pas un nom de classe.
\item \texttt{cannot call this in main}
\newline
Est levée si \texttt{this} est appelé en dehors d'une déclaration de classe.
\item \texttt{left operand not instance of a class}
\newline
Est levée si l'identifier a gauche d'un appel de champ n'est pas l'instance d'une classe.
\item \texttt{no such field in class}
\newline
Est levée si l'identifier \`a droite d'un appel de champ ne correspond pas \`a un champ de la classe \`a gauche.
\item \texttt{identifier is not a field}
\newline
Est levée si l'identifier \`a droite d'un appel de champ correspond \`a une méthode.
\item \texttt{field is protected}
\newline
Est levée si on essaie d'appeler un champ \texttt{protected} en dehors de la classe \`a laquelle il appartient.
\item \texttt{expression not instance of a class}
\newline
Est levée si l'identifier a gauche d'un appel de méthode n'est pas l'instance d'une classe.
\item \texttt{direct method call in main}
\newline
Est levée si une méthode est appelée sans préciser la classe, en dehors d'une déclaration de classe.
\item \texttt{identifier is not a method}
\newline
Est levée si l'identifier a droite d'un appel de méthode correspond a un champ.
\item \texttt{number of parameters does not match signature}
\newline
Est levée si une méthode est appelée avec un nombre de paramètres différent du nombre de paramètre dans sa signature.
\item \texttt{parameter type does not match signature}
\newline
Est levée si le type d'un paramètre dans un appel de méthode ne correspond pas au type de paramètre dans la signature de la méthode.
\item \texttt{class type in call not subclass of class type in signature}
\newline
Est levée si, lorsqu'une méthode demande une instance de classe en paramètre, la classe appelée n'est pas une sous-classe de la classe donnée en signature. 

\end{itemize}


\subsection{Erreurs \`a l'\'execution}

\end{document}
